%\vspace{-4cm}
\chapter*{\color{oxfordblue} Personal Contributions}

The work presented in this thesis is inherently collaborative, having been carried out as a member of the ATLAS Collaboration. This thesis concentrates on the two subjects to which I mostly contributed: the development of new heavy flavour taggers and the combined \vhbc\ analysis using the full Run 2. While I produced some ``ATLAS'' labelled figures, others were taken from public results produced by other members of the Collaboration. Plots without label have been personally produced, with some exceptions in the analysis chapter. This section highlights my personal contributions to these different projects that are fully detailed in this thesis.

\subsubsection{Flavour Tagging}
I joined the flavour tagging group for my qualification task, and have contributed to the training of the new taggers. My main contributions are:
\begin{itemize}
    \item Producing training samples with the new ATLAS software release (R22) for Run 3.
    \item Modifications to the preprocessing to implement importance sampling when harmonising distributions of different flavour samples with the full simulation statistics.
    \item Modifications to the training software to include taus, flexible input variables definition, and general debugging.
    \item Modifications to the postprocessing to implement new visualisation and graphics. 
    \item Retraining DL1r on the new R22 release and comparing it to the previous versions as well as pre-training studies on samples from older software releases. 
    \item First PFlow and variable-radius (VR) training of DL1d with the DIPS sub-tagger.
    \item Hyperparameter and input optimisation of DL1d.
    \item Training the DIPS sub-tagger with VR jets.
    \item Hyperparameter optimisation studies of GN1 and GN2, with modifications to the software stack to leverage the $\mu P$ parametrisation and implement the $\mu$Transfer algorithm.
    \item Adapting the codebase and developing a framework to train on CERN's KubeFlow server. 
\end{itemize}
These different contributions led me to significantly participate in the development of the \textsc{Umami} \cite{UmamiCite} and \textsc{Salt} \cite{SaltCite} software used to train the networks. My contributions have been part of different ATLAS publications, such as Ref. \cite{ATL-PHYS-PUB-2022-027} with a DL1r model I trained, Ref. \cite{ATL-PLOT-FTAG-2023-01} with a DL1d model I trained, Ref. \cite{ATL-PHYS-PUB-2023-021} for which I produced the DL1d input to the $X_{bb}$, as well as an upcoming ATLAS publication on GN2. I led the effort on hyperparameter optimisation of GN2, producing the public results in Ref. \cite{publicplotMUP} presented at the $6^{th}$ Inter-Experimental Machine Learning Workshop \cite{IMLTalk} and the Cloud-Native AI Day KubeCon conference \cite{KubeconTalk}.

\subsubsection{\boldvhbc\ Analysis}
I joined the \vhbc\ analysis team in 2021, and my main contributions are:
\begin{itemize}
    \item Comparing the $X_{bb}$ tagger to DL1r for the boosted \vhb\ by studying the impact on the signal sensitivity with dedicated analysis MVA trainings.
    \item Studies of the Data-Monte Carlo agreement in \vhc\ with DL1r-based tagging. 
    \item Contributing to different rounds of analysis samples productions.
    \item Design and study of a new top control region for the \vhc\ and \vhb\ resolved, with studies leading to the final approach presented in this thesis. Additional study on the Higgs candidate reconstruction strategy in this control region.
    \item Derivation and harmonisation of the \ptv-dependent $\Delta R_{cc}$ cuts in \vhc.
    \item Training and deployment of the CARL models for the single-top $Wt$- and $t$-channels of the top background for the resolved \vhb.
    \item Development of the analysis modelling software to study the top backgrounds. 
    \item Modelling studies of the top background in the resolved \vhb. Derived shape and acceptance uncertainties for \ttb, $Wt$, and $t$-channel, and studied the effect of the chosen modelling and the combination of \ttb\ with $Wt$. 
    \item Numerous fit studies to validate new samples, the new top control region and top backgrounds normalisation scheme, the new Higgs candidate reconstruction strategy, as well as studying the impact of the introduction of CARL models and refinements to modelling. 
    \item Modification to the fit framework to use the new ROOT version and help stabilise the fit, as well as integrating an update to the output results visualisation.
\end{itemize}
At the time of writing, the analysis is reaching its conclusion with final studies on the modelling and the fit framework. It is aiming for publication by June 2024. The results presented here are therefore only temporary and partial as the analysis was still blinded.