\section{Experimental Uncertainties}\label{sec-unc}
While a lot of effort goes into correctly simulating the collection and reconstruction of information, inaccuracies permeate this procedure and must be accounted for in the statistical analysis of Section \ref{sec-fitFramework}. Several types of experimental uncertainties are considered in the analysis, to cover the systematics effects from to the detector performance, the reconstruction of objects, and the effects of flavour tagging. Table \ref{tab:ExpSysts} summarises the different sources of uncertainty, which are further detailed in this section.

\input{Tables/VH/ExpUnc.tex} %TODO check that the ftag uncertainties are still completely decorrelated for the VR track and PFlow

\paragraph{Luminosity \& Pile-up} The measured Run 2 luminosity for ATLAS is 140.1 $\pm 1.2$ fb$^{-1}$ with an uncertainty of 0.83\% \cite{ATLAS:2022hro}. The measurement relies on $x-y$ beam separation scans combined with information from dedicated luminosity-sensitive detectors. The \gls{pu} uncertainty for simulated events is obtained by varying the data rescaling factor of the nominal average pile-up $\langle \mu \rangle$. This factor is introduced due to the observation that \gls{mc} samples match data at a higher $\mu$ than used in their simulation. This rescaling factor is used to reweight the data, matching a simulated-$\mu$ of 1.0 to a data-$\mu$ of 1.09, a rescaling summarised as $1.0/1.09$. A 1$\sigma$ uncertainty on the average \gls{pu} is measured by varying the factor from $1.0/1.0$ to $1.0/1.18$. % WARNING check this is still true

\paragraph{Triggers} Uncertainties on the trigger efficiencies are derived for the electron, muon, and \etm\ triggers. Statistical and systematics effects are combined for the electron trigger uncertainty, while they are considered separately for the muon triggers. Scale factors for the \etm\ trigger efficiency are derived from $W+$jets events, taking into account the statistics of the dataset, assessing systematics effects by deriving scale factors with alternative top and $Z$+jets samples, and modelling the efficiency dependency on the scalar sum of all final state jets.

\newpage
\paragraph{Leptons and $\boldsymbol{E_T^{\text{miss}}}$} Leptons and \etm\ are calibrated in dedicated analyses, with a reduced set of uncertainties propagated here consisting of:
\begin{itemize}[leftmargin=*]
    \item \textit{\etm}: scale factors account for the direction of the \etm\ and the soft term contribution. 
    \item \textit{Electrons}: uncertainties on the reconstructed values, the identification efficiency, isolation efficiency, and the energy scale and resolution are derived by comparing data and simulations in kinematic distributions of $Z \rightarrow e^+ e^-$, $W\rightarrow e\nu$ and $J/\psi \rightarrow e^+e^-$ events \Cite{Aaboud:2657964}. 
    \item \textit{Muons}: uncertainties on the reconstruction and identification efficiencies of muons with $p_T > 15$ GeV and $p_T < 15$ are included separately, using respectively samples of $Z\rightarrow \mu^+\mu^-$ and $J/\psi \rightarrow \mu^+\mu^-$ \cite{Aad:2746302}. Additionally, uncertainties on the isolation efficiency, track-to-vertex association efficiency, momentum scale and resolution as well as charge-dependent misalignment effects are considered. 
    \item \textit{Taus}: hadronically decaying $\tau$-leptons uncertainties on the reconstruction and \gls{rnn}-based identification efficiencies as well as the electron veto efficiencies are derived, from samples of $Z\rightarrow\tau^+ \tau^-$ and top-quark decays to taus \cite{ATL-PHYS-PUB-2019-033, ATL-PHYS-PUB-2015-045, ATLAS-CONF-2017-029}.
\end{itemize}

\paragraph{Jets} are calibrated in dedicated analyses, of which two reduced sets of uncertainties are propagated to the combined \vhbc\ for small- and large-$R$ jets. For the small-$R$ jets, these uncertainties cover \textit{in-situ} analyses, $\eta$-intercalibration, flavour composition, punch-through jets, high-$p_T$ hadrons, and pile-up effects as well as the jet energy scale and resolution measured in data \cite{ATLASjesjerMeas, Aad:2854733}, as described in Section~\ref{sec-atlas-jets}. The reduced set is derived from a \glsfirst{pca} to preserve the largest correlations in certain regions of jet kinematics. Large-$R$ jets uncertainties for the energy scale and resolution are similarly estimated from data \cite{ATLAS:2018bip}. An uncertainty covering the calibration discrepancy between data and \gls{mc}-simulations is also included.

\paragraph{Flavour Tagging} A dedicated calibration is performed to derive flavour tagging scale factors in the resolved regime, as described in Section \ref{sec-selectionandcat}, and the general flavour tagging uncertainties are used for the boosted regime, as described in \ref{chap-calibration}. These flavour tagging calibration \glspl{sf} are derived by combining data-\gls{mc} efficiency modelling \glspl{sf} and \gls{mc}-\gls{mc} \glspl{sf} to account for variations to parton showering and hadronisation. These scale factors are smoothed using a local polynomial kernel estimator to avoid distortions in the kinematic variables \cite{ATL-PHYS-PUB-2020-004}. For each jet flavour, there is one uncertainty per $p_T$ bin in the calibration. A $\tau$-jet uncertainty is derived from the $c$-jet values. \gls{pca} is deployed to reduce the large set of systematics uncertainties to 45 (5) for $b$-jet, 20 (4)for $c$-jet, and 20 (4) for light-jets in the resolved (boosted) regime. Additional uncertainties are added to model to extrapolation of the performance to high-$p_T$ jets. \gls{gnn} truth tagging uncertainties are covered by these flavour tagging uncertainties, so no dedicated uncertainties are considered.

\begin{figure}[h!]
    %\hspace{-2cm}
    \centering
    \makebox[\linewidth][c]{%
        \begin{subfigure}[b]{0.37\textwidth}
            \centering
            \includegraphics[width=\textwidth]{Images/VH/Own_fit/backCom_uncPrefit/GlobalFit_unconditional__Prefit/C_SRCRs_L012_BMin400.png}
            \caption{Boosted regime ($\geq 400$ GeV).}
            \label{fig:backCom_boos}
        \end{subfigure}
        \begin{subfigure}[b]{0.37\textwidth}
            \centering
            \includegraphics[width=\textwidth]{Images/VH/Own_fit/backCom_uncPrefit/GlobalFit_unconditional__Prefit/C_SRCRs_L0_BMax250_BMin150.png}
            \caption{0L, \ptv\ $\in$ [150, 250] GeV.}
            \label{fig:backCom_0L_1}
        \end{subfigure}
        \begin{subfigure}[b]{0.37\textwidth}
            \centering
            \includegraphics[width=\textwidth]{Images/VH/Own_fit/backCom_uncPrefit/GlobalFit_unconditional__Prefit/C_SRCRs_L0_BMax400_BMin250.png}
            \caption{0L, \ptv\ $\in$ [250, 400] GeV.}
            \label{fig:backCom_0L_2}
        \end{subfigure} 
    }\\
    %\hspace{-2cm}
    \makebox[\linewidth][c]{%
        \begin{subfigure}[b]{0.37\textwidth}
            \centering
            \includegraphics[width=\textwidth]{Images/VH/Own_fit/backCom_uncPrefit/GlobalFit_unconditional__Prefit/C_SRCRs_L1_BMax150_BMin75.png}
            \caption{1L, \ptv\ $\in$ [75, 150] GeV.}
            \label{fig:backCom_1L_1}
        \end{subfigure}
        \begin{subfigure}[b]{0.37\textwidth}
            \centering
            \includegraphics[width=\textwidth]{Images/VH/Own_fit/backCom_uncPrefit/GlobalFit_unconditional__Prefit/C_SRCRs_L1_BMax250_BMin150.png}
            \caption{1L, \ptv\ $\in$ [150, 250] GeV.}
            \label{fig:backCom_1L_2}
        \end{subfigure}
        \begin{subfigure}[b]{0.37\textwidth}
        \centering
        \includegraphics[width=\textwidth]{Images/VH/Own_fit/backCom_uncPrefit/GlobalFit_unconditional__Prefit/C_SRCRs_L1_BMax400_BMin250.png}
        \caption{1L, \ptv\ $\in$ [250, 400] GeV.}
        \label{fig:backCom_1L_3}
        \end{subfigure} 
    }   \\
    %\hspace{-2cm}
    \makebox[\linewidth][c]{%
        \begin{subfigure}[b]{0.37\textwidth}
            \centering
            \includegraphics[width=\textwidth]{Images/VH/Own_fit/backCom_uncPrefit/GlobalFit_unconditional__Prefit/C_SRCRs_L2_BMax150_BMin75.png}
            \caption{2L, \ptv\ $\in$ [75, 150] GeV.}
            \label{fig:backCom_2L_1}
        \end{subfigure}
        \begin{subfigure}[b]{0.37\textwidth}
            \centering
            \includegraphics[width=\textwidth]{Images/VH/Own_fit/backCom_uncPrefit/GlobalFit_unconditional__Prefit/C_SRCRs_L2_BMax250_BMin150.png}
            \caption{2L, \ptv\ $\in$ [150, 250] GeV.}
            \label{fig:backCom_2L_2}
        \end{subfigure}
        \begin{subfigure}[b]{0.37\textwidth}
            \centering
            \includegraphics[width=\textwidth]{Images/VH/Own_fit/backCom_uncPrefit/GlobalFit_unconditional__Prefit/C_SRCRs_L2_BMax400_BMin250.png}
            \caption{2L, \ptv\ $\in$ [250, 400] GeV.}
            \label{fig:backCom_2L_3}
        \end{subfigure} 
    }
    \caption{The background composition of the different analysis regimes and lepton channels, with the data - Monte Carlo prefit agreement displayed in the bottom panels.}
    \label{fig:backCom}
\end{figure} 

\section{Signals and Backgrounds Modelling}\label{sec-mod}
Similarly to the experimental process, the simulations of the signals and backgrounds cannot entirely be accurate and mismodellings are to be expected in the derived samples. These inaccuracies must be taken into account in the fit framework to avoid introducing bias. The modelling strategy of the signals and backgrounds at the time of writing is discussed in this section. The background composition depends on the lepton channel, the analysis category, and the \ptv\ and \nj, as highlighted in Figure \ref{fig:backCom}. The $V$+jets processes are the dominant backgrounds in the signal regions of the 0-lepton and 2-lepton channels, while the top processes contribute more in the 1-lepton channel, and globally at larger jet multiplicities and lower \ptv. From the flavour tagging requirements, \vhb\ primarily selects the $bb$-component of the background while \vhc\ has a more diverse flavour composition with the 2 $c$-tag as an intermediate step between the $BB$ and 1 $c$-tag. This translates into increased fractions of \vhf\ in \vhb, and \vmf\ and \vlf\ in \vhc. To summarise the largest backgrounds per lepton channels:
\begin{itemize}[leftmargin=*]
    \item \textbf{0-lepton}: the dominant background is the $Z$+jets with a sizeable $W+$jets component, particularly in \vhc\ due to large \etm\ or miss-identified hadronic $\tau$. In \vhb, the top background significantly contributes and dominates in 3- and 4-jets. Finally, there is some diboson contribution, primarily for $BB$-tagged events.
    \item \textbf{1-lepton}: the top process is dominant for \vhb, while for \vhc\ the $W+$jets leads followed by the top and the multi-jet backgrounds.  
    \item \textbf{2-lepton}: $Z+$jets makes up most of the background, followed by the diboson and some residual top process at low \ptv\ for \vhb. 
\end{itemize}
The changing background composition in the different analysis regions requires an adequate strategy to constrain their modelling in the fit, as outlined in this section.
  
\subsection{General Modelling Strategy}\label{sec-modStrat}
The combined analysis adopts some common strategies to model the signals and backgrounds that are described in this section, before reviewing the specificities adopted for each process. A guideline for the modelling is to treat backgrounds coherently across analysis regimes and correlate uncertainties between \vhb\ and \vhc\ when possible. The normalisations of the major backgrounds, the $V+$jets and Top, are free to float in the fit, with \glspl{fn} split by \ptv\ and jet multiplicity when the statistics allow. Minor backgrounds are fixed at \gls{mc} predictions with a normalisation uncertainty. To account for \gls{mc}-generator modelling uncertainties, comparisons of the nominal samples to the alternative samples introduced in section \ref{sec-datasets} and summarised in Table \ref{tab:summary_altsamples} are performed. For each process, the uncertainties are split into normalisation, relative acceptance, and shape uncertainties.  % Check top s is treated like that;

\begin{table}[!h]
    \centering
    \renewcommand{\arraystretch}{1.1}
    \begin{tabular}{llll}
      \hline \hline 
      \textbf{Sample} & \textbf{Nominal Generator} & \textbf{Alternative Generators} & \textbf{Systematics Effects} \\
      \hline
      \vhb\ & \textsc{Powheg} + \textsc{Pythia 8} & \textsc{Powheg} + \textsc{Herwig 7} & $\mu_R$, $\mu_F$, \gls{isr}, \gls{fsr}, \gls{pdf}\\
      \hline
      \vhc\ & \textsc{Powheg} + \textsc{Pythia 8} & \textsc{Powheg} + \textsc{Herwig 7} & $\mu_R$, $\mu_F$, \gls{isr}, \gls{fsr}, \gls{pdf} \\
      \hline
      $V$+jets & \textsc{Sherpa} 2.2.11 & \textsc{MadGraph5 FxFx}, & $\mu_R$, $\mu_F$, \gls{pdf}, \\
                                            & & \textsc{Sherpa} 2.2.1 & EW corrections \\
      \hline
      \ttb\ and & \textsc{Powheg}+\textsc{Pythia} 8 & \textsc{Powheg}+\textsc{Herwig} 7,  & \gls{isr}, \gls{fsr}, \\
      single-top &  & \textsc{MadGraph5}+\textsc{Pythia} 8  & DS/DR (for $Wt$) \\
      \hline
      Diboson & \textsc{Sherpa} 2.2.11  & \textsc{Powheg}+\textsc{Pythia} 8, & $\mu_R$, $\mu_F, \gls{pdf}$,\\
       &  & \textsc{Sherpa} 2.2.1 & \gls{ew} corrections\\
      \hline \hline 
    \end{tabular}
    \caption{Summary of nominal and alternative samples in the analysis. Alternative samples include different generator and systematics effects from modification to the nominal setup.}
    \label{tab:summary_altsamples}
\end{table}
  
\paragraph{Normalisation uncertainties} are overall uncertainties on the yield of a process, computed in and applied to all regions. These uncertainties are considered from expected yield of a background to derive its normalisation from data, for the diboson and single-top $s$ processes primarily.

\paragraph{Acceptance uncertainties}: relative acceptance uncertainties cover possible changes in the distribution of events of a specific process across the different regions of the analysis phase space. They account for the migration of events between these regions and are assessed by measuring the change in the ratio of events between regions when switching to differently generated samples (indexed by $i$ here). The priors on these uncertainties are calculated with the \textit{double ratio} 
\begin{equation}\label{eq-doubleRatio}
    \text{Acceptance Unc}_i = \frac{\text{Yield}[\text{Cat.}^B (\mathrm{Alternative}_i\mathrm{\,\,MC})]}{\text{Yield}[\text{Cat.}^A (\mathrm{Alternative}_i\mathrm{\,\,MC})]} \Bigg/ \frac{\text{Yield}[\text{Cat.}^B (\mathrm{Nominal\,\,MC})]}{\text{Yield}[\text{Cat.}^A (\mathrm{Nominal\,\,MC})]},
\end{equation}
where category $A$ ($\text{Cat.}^A$) is the region with the highest purity in the studied process, and $B$ ($\text{Cat.}^B$) is the region extrapolated to. If several alternative generators are used ($i > 1$), their respective double ratios are summed in quadrature: \[ \text{Total Acceptance Unc} = \sqrt{\sum_i\left(\text{Acceptance Unc}_i\right)^2}.\] If the extrapolation is across several regions $A$, $B$, $C$ ordered by decreasing purity, the acceptance ratio is decomposed into two extrapolations: a first one from $A \rightarrow B+C$ followed by an additional $B \rightarrow C$ uncertainty. For acceptance uncertainties between distinct analysis regions in the resolved regime, the signal and Top $BT$ control regions are considered jointly due to their similar kinematics. The acceptance uncertainties between these two regions are modelled by the flavour tagging uncertainties.

\paragraph{Shape uncertainties}: the \glspl{bdt}, $m_{bb}$, $m_{cc}$, and \ptv\ shapes of the processes in the different regions are given some flexibility in the fit by introducing shape uncertainties derived from a comparison of the nominal to the alternative samples. The combined analysis introduces the novel \gls{carl} technique to derive a reweighted shape uncertainty using a neural network \cite{carl}. A \gls{dnn} is trained to discriminate nominal events from alternative ones, with the process repeated for each alternative sample. The output of the \gls{carl} network is a score representing the probability for an event to belong to the alternative sample. This is used to reweight the nominal distribution into the alternative distribution, analogously to truth tagging. The advantage of this technique is that the reweighted nominal distributions benefit from a much larger statistics than the alternative ones, thus smoothing out intra-bin fluctuations and reducing the \gls{mc} statistics uncertainties. Examples of such derived \gls{carl} shape uncertainties modelling the parton shower with \textsc{MadGraph5\_aMC@NLO} for the single-top $Wt$ process in 1-lepton are presented in Figure \ref{fig:carl:resolved_closure_stopWt}. Additional shape uncertainties are directly derived by comparing samples for \gls{ew} corrections, \gls{qcd} scales, $V+$jets and diboson \ptv\ modelling with \textsc{Sherpa} 2.2.1, parton shower alternative for the signal samples, and uncertainties for the single-top $Wt$ DS / DR shapes.

\begin{figure}[!htbp]
    \centering
      \subfloat[SR BDT distribution.]{
        \includegraphics[width=0.48\textwidth]{Images/VH/Carl/wt/sr.png}
      }
      \subfloat[\highdr\ CR $m_{bb}$ distribution.]{
        \includegraphics[width=0.48\textwidth]{Images/VH/Carl/wt/crhigh.png}
      }
      \caption{CARL closure plots, between the nominal \textsc{Powheg}\textsc{Pythia}8 (\textit{PwPy8}, with the DR scheme) and the alternative \textsc{MadGraph5\_aMC@NLO} (\textit{aMCatNLO}), for the single-top $Wt$ production in \vhb, 1-lepton, 75 GeV < \ptv\ < 400 GeV, and 2 jets. The CARL interpolation (orange) of the nominal (blue) into the alternative (red) is smoother and with lower MC-stats. uncertainty. The top plots show the distributions, the middle plots the ratios, and the bottom plots the residuals.}
      \label{fig:carl:resolved_closure_stopWt}
  \end{figure}
  
\input{Tables/VH/ModUncSum.tex} % WARNING, need to check the signal uncertainties

\paragraph{}An overview of the signals and backgrounds modelling systematics considered is presented in Figure \ref{tab:syst_summary} and detailed in Appendix \ref{appsec-vh-backsigmod}. All uncertainties presented here are further processed before entering the fit. To remove large statistical fluctuations potentially present in shape systematics, these shapes are smoothed by iteratively rebinning the distribution until the statistical uncertainty in each merged bin of the nominal distribution is smaller than~5\%. If a systematics has a negligible impact on the distributions in the fit, it is pruned away to ease convergence and reduce the fit complexity. This is applied to systematics causing a normalisation effect smaller than~0.5\% or when both the up- and down-variations have the same sign. Shape uncertainties are pruned if no bin in the distribution has a deviation above 0.5\% after the overall normalisation, or if only one of the up- or down-variation is non-zero. For very small background processes, both shape and normalisation uncertainties are pruned: if this is a signal-sensitive region, when the signal yield is > 2\% of the total in the region, the uncertainties are pruned if the process is $\leq$ 2\% of the signal. In non-signal sensitive regions, the process must be $\leq$ 0.5\% of the total background to be pruned. The rest of this section goes into the details of the modelling, highlighting some specificities and subtleties related to each process. 

\subsection{Signal Modelling}\label{sec-modSignal}
The three main signal productions $qq \rightarrow WH$, $q\bar{q} \rightarrow ZH$, and $gg \rightarrow ZH$ are modelled separately, with uncertainties addressing the production and the decay mode of the Higgs into $b\bar{b}$ or $c\bar{c}$. The goal of the analysis is to measure the fiducial cross-sections of the \vhb\ and the signal strength of the \vhc. This first objective is approached with the adoption of the \glsfirst{stxs} in the reduced scheme of stage 1.2 \cite{badger2016les, berger2019simplified}, depicted in Figure \ref{fig:model-stxsscheme}. The bins are defined in successive regions of \ptv, from truth information in the simulated samples, and the number of additional jets in the event, at 0 or more than 1 additional jet.
  
\begin{figure}[!htbp]
    \centering
    \includegraphics[width=0.58\textwidth]{Images/VH/Model/STXSsketch.png}
    \caption{The Standard Template Cross-Section scheme in the reduced stage 1.2 \cite{berger2019simplified}.}
    \label{fig:model-stxsscheme}
\end{figure}

\paragraph{}The signal samples are finely binned following the \gls{stxs} prescription, with 5 \ptv\ bins for the $ZH$ covering $[75, 150[$ GeV, $[150, 250[$ GeV, $[250, 400[$ GeV, $[400, 600[$ GeV, and $\geq$ 600 GeV. The first three bins, corresponding to the resolved regime, are further split  with 0 or $\geq 1$ additional jet, for a total of 8 different \glspl{poi} measured in $ZH$. For $WH$, the binning is similar to $ZH$ but there is no jet multiplicity split in the $[75, 150[$ GeV bin, giving a total of 7 \glspl{poi} for $WH$. The full \gls{stxs} categorisation is used for \vhb\ and also for the \vhc, to enable correlation of the $VH$ uncertainties. For the \vhc, the templates are later merged and only one \gls{poi} is extracted: the global signal strength.\\
  
The signal is coherently modelled across the resolved and boosted regimes and targeted final state \vhb\ or \vhc. Several uncertainties are implemented to model the $VH$ production of the $H \rightarrow b\bar{b}/c\bar{c}$ decay. These uncertainties include:
\begin{itemize}[leftmargin=*]
    \item \textit{\gls{qcd} scale uncertainties}: obtained by varying the renormalisation and factorisation scales $\mu_R$ and $\mu_F$. These variations are the most impactful in the theoretical prediction of the $VH$ production cross-sections. They are considered as shape uncertainties, implemented to cover modifications to the inclusive cross-sections and to parametrise possible migrations across \ptv\ and additional jet multiplicity bins, following Ref. \cite{ATL-PHYS-PUB-2018-035}. The quark- and gluon-initiated signal processes have cross-section modifications parametrised separately. 
    \item \textit{\gls{pdf} + $\alpha_s$ uncertainties}: alternative parton distributions from the \textsc{PDF4LHC15\_30} modifying the $VH$ cross-sections in \gls{stxs} bins are considered  \cite{Butterworth:2015oua}. The $VH$ cross-sections in each \gls{stxs} bin are systematically modified by comparing the nominal \gls{pdf} to 30 alternatives. Furthermore, the $\alpha_s$ estimated at the $Z$ mass is varied for the nominal setup within its uncertainties. These uncertainties are separately calculated for $qq$-initiaded $WH$ and $ZH$, and for $gg$-initiated $ZH$. Shape effects on the resolved regime \ptv\ distributions are considered, while variations to the boosted large-$R$ mass $m_J$ and the invariant mass $m_{bb}$ or $m_{cc}$ are negligible.
    \item \textit{\gls{ew} corrections}: NLO electroweak corrections from NNLO \gls{ew} effects are considered with uncertainties modifying the \ptv\ distributions.
    \item \textit{Branching ratio}: a theoretical uncertainty of 1.61\% on the $H \rightarrow{b\bar{b}}$ branching ratio and an uncertainty covering the range from -1.99\% to +5.53\% for the $H \rightarrow{c\bar{c}}$ branching ratio are considered \cite{LHCHiggsCrossSectionWorkingGroup:2016ypw}. The $ZH$ ($WH$) cross-sections cover 96.52\% to 104.11\% (97.95\% to 101.98\%) of their values thanks to additional uncertainties.
    \item \textit{Parton shower and underlying event uncertainties}: variations to the \gls{ps} and \gls{ue} can affect the properties of the $H \rightarrow b\bar{b} / c\bar{c}$ decays. Uncertainties are introduced to model these effects on signal acceptance. In the resolved regime, the effects of an alternative \gls{ps} model on the signal acceptance are evaluated on truth information in a similar phase space to the analysis selection. Acceptance uncertainties are derived by comparing the signal acceptance in the analysis categories between the nominal \textsc{Pythia} 8 and the alternative \textsc{Herwig} 7. Additional sub-leading acceptance uncertainties are evaluated by modifying the \textsc{Pythia} AZNLO tune. Differences in \ptv\ and $m_{bb}$ ($m_{cc}$) between \textsc{Pythia} and \textsc{Hewrig} are also considered, and the shape difference in the \gls{mva} distribution when adopting \textsc{Powheg}+\textsc{Herwig} 7 is used in the final stage of the analysis. In the boosted regime, the same strategy with the same \gls{ps} models is adopted, but the full detector response and event reconstruction are simulated with uncertainties covering modifications to the $m_J$ distributions.
\end{itemize}

\subsection[$V+$jets Modelling]{$\boldsymbol{V+}$jets Modelling}\label{sec-modVjet}
The $V+$jets processes are modelled separately for $Z+$jets and $W$+jets, depending on the flavour of the reconstructed vector boson. Their modelling nonetheless shares many similarities.

\subsubsection{$\boldsymbol{Z+}$jets}
This background is dominant in the 0L and 2L channels and limited in 1L. Different components are split based on the flavour composition of jets selected to form the Higgs candidate, grouping compositions with similar kinematic performance as:   
\begin{itemize}
    \item \textit{$Z+$ heavy flavours (\zhf)}: $Z+bb$ and $Z+cc$.
    \item \textit{$Z+$ mixed flavours (\zmf)}: $Z+bc$, $Z+bl$, and $Z+cl$.
    \item \textit{$Z+$ light flavours (\zlf)}: $Z+l$.
\end{itemize}
Each grouping has its own free-Floating Normalisations (\glspl{fn}) in 0L and 2L, with \zhf\ dominant in \vhb\ and the other two components significant in \vhc. These \glspl{fn} are decorrelated in \ptv\ and total jet multiplicities \nj\footnote{Except for the 2L with 75 GeV < \ptv\ < 150, where the \vhb\ 4-jet is merged with 3-jet but \vhc\ is not: to account for this, \vhb\ has an extra \zhf\ \gls{fn} for 3p-jet.}. The modelling of $Z+$jets includes several types of acceptance uncertainties that are applied only in 0L and 2L. In the resolved regime:
\begin{itemize}[leftmargin=*]
    \item \textit{Channel extrapolation 2L $\rightarrow$ 0L uncertainties}: for the \zhf, \zmf, and \zlf\ separately. 
    \item \textit{Flavour composition uncertainties}: accounting for the variation on the yields of different flavours in the combinations with the double ratio of Equation \ref{eq-doubleRatio}. These include a ratio of $cc$ to $bb$ for \zhf, and of $bc$ and $bl$ to $cl$ for \zmf. They are decorrelated in \ptv\ and jet multiplicity \nj\ bins. 
    \item \textit{Region extrapolation uncertainties}: are included to model the acceptance of different regions, and derived with the double ratio Equation \ref{eq-doubleRatio} from a high purity region to a lower purity as:
    \begin{itemize}
        \item \zhf\ and \zmf: constrained mostly in the CRHigh and applied to the \gls{sr}. 
        \item \zlf: constrained mostly in 1 $LN$-tagged $V+l$ CR and the \gls{sr}, thus applied in CRHigh. % WARNING, in the text, it says that it's SR - CRHigh, but it's constrained in LN??
    \end{itemize}
\end{itemize}
The values of the acceptance uncertainties are presented in Table \ref{tbl:zjets_acc_full} of Appendix \ref{appsec-vh-backsigmod}. In addition, 4 different types of shape uncertainty are considered:
\begin{itemize}
    \item \gls{carl} shape: modelling the difference between \textsc{Sherpa} 2.2.11 and \textsc{MadGraph FxFx}, derived for all components and applied in all analysis regions.
    \item \textsc{Sherpa} 2.2.1 \ptv\ shape uncertainties to model the data-\gls{mc} mismodelling of \ptv\ in \textsc{Sherpa} 2.2.11. 
    \item \gls{qcd} scale shape uncertainties by varying $\mu_R$ and $\mu_F$.
    \item \gls{ew} shape variations, although they are typically small.
\end{itemize} 

\paragraph{Boosted regime} The modelling strategy is roughly the same as in the resolved regime, with the uncertainties fully detailed in Appendix Table~\ref{tbl:zjets_acc_fullBoos}. The \zhf\ component is left free-floating in 0L and 2L, while the \zmf\ and \zlf\ components both have overall acceptance uncertainties of 35\%. The \zlf\ has no other acceptance uncertainty since it is negligible in the boosted regime. Flavour acceptance uncertainties for \zhf\ and \zmf\ are applied in 0L and 2L. They also have channel acceptance uncertainties and \gls{sr} $\rightarrow$ Top CR acceptance ratios, both applied in 0L. Additional \ptv\ extrapolation uncertainties from [400, 600] GeV to $> 600$ GeV are considered in 0L and 2L. Shape uncertainties are derived similarly to the resolved regime. 

\subsubsection{$\boldsymbol{W+}$jets}
This background is dominant in the 1-lepton channel, with a residual contribution in 0-lepton due to hadronically decaying $\tau$-lepton. It is split equivalently to the $Z+$jets background as:   
\begin{itemize}
    \item \textit{$W+$ heavy flavours (\whf)}: $W+bb$ and $W+cc$
    \item \textit{$W+$ mixed flavours (\wmf)}: $W+bc$, $W+bl$, $W+b\tau$, $W+cl$, and $W+c\tau$.
    \item \textit{$W+$ light flavours (\wlf)}: $W+l$, $W+l\tau$, $W+\tau\tau$.
\end{itemize}
Each grouping has its own floating normalisation, with \whf\ significant in \vhb, while \wmf\ and \wlf\ are more important in \vhc. The \glspl{fn} are decorrelated in \ptv\ and jet multiplicities \nj\footnote{The only exception is the 1L \wlf\ in 75 GeV $<$ \ptv\ $<$ 150 GeV: it has a 25\% normalisation uncertainty.}. Acceptance uncertainties, listed in the Appendix Table \ref{tbl:wjets_acc_full}, are applied in 0L and 1L. They include:
\begin{itemize}[leftmargin=*]
    \item \textit{Channel extrapolation 1L $\rightarrow$ 0L uncertainties}: applied in 0L for all components separately. 
    \item \textit{Flavour composition uncertainties}: include a comparison of $cc$ to $bb$ for \whf, of [$bc$, $bl$, $c\tau$, $b\tau$] to $cl$ for \wmf, and of [$l\tau$, $\tau\tau$] to $l$ for \wlf. They are decorrelated in \ptv\ and \nj.
    \item \textit{Region extrapolation uncertainties} are defined differently for the combinations:
    \begin{itemize}
        \item \whf: constrained mostly in the \gls{sr} and the $BB$-tagged CRLow\footnote{\label{footnote-crlow}The CRLow is always only considered in \vhb\ 1L.}, applied to CRHigh in different \ptv\ regions. For \vhb\ 1L, an extra CRLow $\rightarrow$ SR is applied. 
        \item \wmf: constrained mostly in 2 $c$-tagged CRHigh, applied in SR and CRLow\cref{footnote-crlow}.
        \item \wlf: constrained mostly in the SR and the 1 $LN$-tagged $V+l$ CR, applied in CRHigh. 
    \end{itemize}
    \item \textit{\nj\ acceptance}: \glspl{fn} are left free-floating in \nj\ for 2- and 3-jet. For \vhb, the 4-jet category has no dedicated \gls{cr} and a 3-jet $\rightarrow$ 4-jet acceptance is applied to \whf.
\end{itemize}
In addition, 4 different types of shape uncertainties are considered similarly to the $Z+$jets.

\paragraph{Boosted regime} The same modelling strategy as $Z+$jets is applied, with the uncertainties fully detailed in Appendix Table \ref{tab:wjets_acc_fullBoos}. The \whf\ component is left free-floating in 0L and 1L, while the \wmf\ and \wlf\ components have overall acceptance uncertainties of 36\% and 38\% respectively. Flavour acceptance uncertainties are considered for \whf\ from $bb$, and for \wmf\ from $bc$. The different components have channel acceptance uncertainties applied in the 0L channel and \gls{sr} $\rightarrow$ Top CR acceptance ratios applied in the 0L and 1L channels. Additional \ptv\ extrapolation uncertainties from [400, 600] GeV to $> 600$ GeV are considered in 0L and 1L. Shape uncertainties are derived similarly to the resolved regime. 

\subsection{Top Modelling}\label{sec-modTop} 
The backgrounds including the decay of a top-quark $t$ are considered here, distinguishing between the \ttb\ pair-production and the single-top $Wt$ production as well as the single-top $t$- and $s$-channels, by decreasing order of relative importance. The \ttb\ and single-top $Wt$ are combined into a unified \textit{Top} component\footnote{Throughout this chapter, Top will refer to the combination of the \ttb\ \& $Wt$ processes.} in the resolved regime, and the single-top $t$- and $s$-channels are considered separately. The Top backgrounds in 0L and 1L are estimated from \gls{mc} and dedicated Top $BT$ control region, with the 2L case described later in this section. In the resolved regime, the Top is grouped into different components based on three truth flavour categories:
\begin{itemize}
    \item Top$(bb)$: which is mostly found in the \vhb\ phase space of the signal regions and the \highdr\ \glspl{cr} due to the large angle between the emitted top-quark that is passed to the $b$-quarks. 
    \item Top$(bq)$: combining Top$(bc)$ and Top$(bl)$, is mostly in the \vhc\ phase space and is well selected by the Top $BT$ \gls{cr}.
    \item Top$(qq)$: combining Top$(cc)$, Top$(cl)$ and Top$(ll)$, where $l$ is a light-jet ($u$, $d$, $s$, a gluon, or a $\tau$), is mostly in the $NT$ and $LT$ regions of the \vhc. 
\end{itemize}

\begin{figure}[!htbp]
    \centering
      \subfloat[]{
        \includegraphics[width=0.48\textwidth]{Images/VH/Model/Top/TopcompoSRNT.png}
      }
      \subfloat[]{
        \includegraphics[width=0.48\textwidth]{Images/VH/Model/Top/TopcompoSRLT.png}
        }
      \caption{The MVA distributions of the top background components (direct tagged) in the \vhc\ signal regions ($NT$-tagged on the left, $LT$-tagged on the right) with 150 GeV $<$ \ptv\ $<$ 250 GeV and 3 jets, before rebinning. Top$(bb)$ in black, Top$(bc)$ in red, Top$(bl)$ in blue, and Top$(qq)$ in green. The bottom panels show the normalised distributions.} 
      \label{fig:topflavdistr_VHcc}
\end{figure}

These groupings are based on the shared kinematics of the components, where the selected jets are either both $b$-jets and thus likely to directly come from the top decays ($bb$), 1 $b$-jet likely from a top decay and 1 non $b$-jet from a subsequent hadronic $W$ decay or a radiated jet ($bc$ and $bl$, summarised $bq$), or neither directly from the top decay ($cc$, $cl$, and $ll$, summarised $qq$). The $bc$ and $bl$ are combined into a single Top$(bq)$ component because they share the same kinematics, as illustrated in Figures \ref{fig:topflavdistr_VHcc} in the signal regions of \vhc. The Top$(bq)$ background is particularly important in \vhc\ as it peaks near the signal mass (having a mass $\sim (m_t + m_W) / 2 \approx m_H$) and therefore exhibits signal-like properties such as reaching high MVA scores, as shown in Figure \ref{fig:topflavdistr_VHcc}. Due to the small contribution of the Top$(qq)$ component, it is merged with the Top$(bq)$ into a single Top$(bq/qq)$ component, with the different subcomponents shapes modelled by flavour composition uncertainties. This section details the modelling of the Top backgrounds in the analysis regimes for 0L and 1L, followed by the single-top $t$- and $s$-channels in resolved, and finally the modelling adopted for the boosted regime.

\subsubsection{The $\boldsymbol{t\bar{t}}$ and $\boldsymbol{Wt}$ in 0L \& 1L Resolved Modelling}
There are three main elements in the Top background modelling scheme in the 0L and 1L resolved regime: floating normalisation, acceptance uncertainties, and shape uncertainties. On the first point, free-floating normalisations are applied for the Top$(bb)$ and the Top$(bq/qq)$ components, constrained primarily in the $BB$-tagged \highdr\ CR and the Top $BT$ \gls{cr}. These \glspl{fn} are separated in jet multiplicity \nj\ as well as \ptv, for a total of 16 \glspl{fn}. Concerning the second point, several types of acceptance uncertainties are applied, as summarised in Table \ref{tab:summary_altsamples} and detailed in the Appendix Table \ref{tab:top_summary}:
\begin{itemize}[leftmargin=*]
    \item \textit{Channel extrapolation 1L $\rightarrow$ 0L uncertainties}: the Top is dominant in 1L, hence the \glspl{fn} derivation is driven by the 1-lepton channel and applied to the 0L. This uncertainty is split in \ptv, with 2\% in [150, 250] GeV and 8\% in [250, 400] GeV.
    \item \textit{Flavour composition uncertainties}: the Top$(bq/qq)$ includes differently shaped subcomponents. Uncertainties are derived from the alternative samples as double ratios comparing the $bl$ (5\%) and $qq$ (10\%) to the $bc$. 
    \item \textit{Region extrapolation uncertainties}: the Top$(bb)$ is dominant in the CRHigh while the Top$(bq/qq)$ leads in the Top $BT$ \gls{cr}, hence the extrapolations differ for the components. They are all derived with double ratios from alternative samples.
    \begin{itemize}
        \item Top$(bb$): extrapolation uncertainties are derived from the CRHigh and applied in the \gls{sr}, the Top \gls{cr} and the CRLow\cref{footnote-crlow}. Additional uncertainties are applied from the \gls{sr} to the Top \gls{cr} and CRLow\cref{footnote-crlow}. All uncertainties are split in \ptv.
        \item Top$(bq/qq)$: the uncertainties are derived from the \gls{sr} + Top \gls{cr} + CRLow\cref{footnote-crlow}, due to their shared kinematic, and applied to the CRHigh. Additional uncertainties are applied from the \gls{sr} + Top \gls{cr} to the CRLow\cref{footnote-crlow}. All uncertainties are split in \ptv.
    \end{itemize}
    \item \textit{Process acceptance ratios}: in the \ttb\ and $Wt$ combination, the \ttb\ dominates and drives the normalisation. Additional acceptance uncertainties are included and applied to the $Wt$ to model differences in the relative contributions of the two processes. These are calculated with a double ratio in the different \ptv\ regions, lepton channels, and flavour components. They range from 12\% to 48\%.
\end{itemize}
In addition, several shape uncertainties are considered for the Top backgrounds: 
\begin{itemize}[leftmargin=*]
    \item \gls{carl} shapes: modelling the difference between the nominal samples (\textsc{Powheg}+\textsc{Pythia} 8) and the alternative modelling of the parton shower (\textsc{Powheg}+\textsc{Herwig} 7) and matrix element (\textsc{MadGraph5\_aMC@NLO}+\textsc{Pythia} 8). These \gls{carl} models are trained separately for \ttb\ and $Wt$ and per lepton channel, inclusively in flavour compositions and \nj. The DR scheme is used as nominal for these training of $Wt$ because the alternative samples use the same \ttb\ overlap removal scheme. 
    \item A DS-DR shape uncertainty is derived uniquely for $Wt$ to account for possible shape effects from modifications to the overlap removal procedure from \ttb. The \textsc{Powheg}+\textsc{Pythia} 8 samples with DS scheme are directly used in the fit as templates, thanks to their sufficient statistics. This shape uncertainty is unique in the analysis as a normalisation uncertainty is simultaneously applied to account for the different yields of the DS- and DR-schemes.
    \item \gls{isr} and \gls{fsr} shape uncertainties are derived by varying the $\mu_R$ and $\mu_F$ scales. Up- and a down-variations are considered for each of them, with symmetric variations for the \gls{isr} while the down-variation of \gls{fsr} is smaller than its up-variation.
\end{itemize} 

\subsubsection{The Single-Top $\boldsymbol{t}$- \& $\boldsymbol{s}$-channels in 0L \& 1L Resolved Modelling}
The single-top $t$- and $s$-channels are almost negligible in the analysis, except in the \vhb\ resolved at low \ptv, where the $t$-channel reaches a total backgrounds fraction of $\sim$8\% in the 1L channel. The importance of single-top $t$ quickly reduces with increasing energy\footnote{Except in the CRHigh region where the ratio stays in the 7\%-9\% range.}. In 0L and 1L, the single-top $t$- and $s$-channels are only applied cross-sections uncertainties of 17\% and 4.6\%, respectively. The single-top $t$-channel has several additional acceptance uncertainties derived by double ratio computations with alternative samples to model: 
\begin{itemize}[leftmargin=*]
    \item \textit{channel extrapolation uncertainty}: of 6\% from 1L to 0L.
    \item \textit{Region extrapolations uncertainties}: depend on the \ptv. For \ptv\ < 150 GeV, the uncertainty is applied from SR $\rightarrow$ CRLow+CRHigh, with an additional CRHigh $\rightarrow$ CRLow uncertainty in 1L. For the higher \ptv\ regions, the extrapolations are instead from CRHigh $\rightarrow$ SR+CRLow\cref{footnote-crlow}, with an additional SR $\rightarrow$ CRLow\cref{footnote-crlow} uncertainty.
    \item \textit{\nj\ acceptance}: are considered from the 3-jet to the 2-jet, and from the 2+3-jet to the 4-jet in 0L.
    \item \textit{\ptv\ extrapolation uncertainties}: since the single-top $t$-channel is mostly present in the lowest \ptv\ regions, \ptv\ extrapolation uncertainties are included from [75, 150] GeV to [150, 400] GeV, with additional [150, 250] GeV to [250, 400] GeV uncertainties.
\end{itemize}
In addition, \gls{carl} and \gls{isr}/\gls{fsr} shape uncertainties are considered for the single-top $t$-channel in 1L only, as is done for the Top background. Table \ref{tab:stopt_summary} of the Appendix details the various single-top uncertainties considered.

\subsubsection{Top Backgrounds in 2L Resolved Modelling} 
Data-driven estimates are used for the 2L channel in the resovled regime only. In \vhb, templates are derived in the Top $e\mu$ region for the Top background with an 0.8\% extrapolation uncertainty to the signal region. For \vhc, the Top $e\mu$ region is used as a control region to let the Top background left free-floating, with at least one tight $c$-tagged required.

\subsubsection{Top Backgrounds Boosted Modelling} 
In the boosted regime, the \ttb\ benefits from a good Top \gls{cr} and is not combined with the $Wt$ in the presented results\footnote{Studies are, at the time of writing, ongoing to merge these two processes in the boosted regime.}. The modelling in the boosted regime, detailed in the Appendix Tables \ref{tab:ttbar_summary_boosted} and \ref{tab:stopt_summary_boosted}, covers:
\begin{itemize}[leftmargin=*]
    \item \ttb: 1 \gls{fn} per \ptv\ region for 0L and 1L, and a 20\% normalisation uncertainty is applied in 2L. \textit{Channel extrapolation uncertainties} are split per \ptv\ and derived from 1L $\rightarrow$ 0L. \textit{Region extrapolation uncertainties} of 10\% are applied in 0L and 1L from the Top \gls{cr} to the \gls{sr}.
    \item Single-top $Wt$-, $t$-, and $s$-channels are not free-floated but insead have respectively 25\%, 10\%, and 4.6\% normalisation uncertainties. The $Wt$ has acceptance uncertainties to cover the lepton channel extrapolation and \ptv\ extrapolation from [400, 600] GeV to $>$ 600 GeV. 
\end{itemize}
Boosted shape uncertainties are considered similarly to what is done in the resolved regime.

\subsection{Diboson Modelling}
The diboson production backgrounds consist of the $WW$, $WZ$, and $ZZ$ processes. In \vhb, the $ZZ$ primarily contributes to the 2L channel, while $WZ$ with $W$ leptonically and $Z$ hadronically decaying contributes to the 1L. Both equally contribute to 0L. \\

In \vhc, the main contributor to 2L is the $WZ$ with the $W$ hadronically decaying and the $Z$ leptonically decaying, while in 1L the $WW$ process contributes the most. Again, both contribute similarly to 0L. The resolved and boosted acceptance uncertainties are detailed in Table \ref{table:VV_Sys_Summary} and Table \ref{table:VV_SysBoos_Summary}. \\

In the resolved regime, the diboson processes are a small background in the analysis, so only normalisations uncertainties are used for $ZZ$ (17\%), $WW$ (16\%), and $WZ$ (19\%) for the $qq$-initiated, and (30\%) for the $gg$-initiated $ggVV$. Uncertainties are correlated between \vhb\ and \vhc. The $VZ (\rightarrow b\bar{b})$ and $VZ (\rightarrow c\bar{c})$ are considered as signals of the cross-check analysis, and denoted as $VZbb$ and $VZcc$ throughout the modelling. The rest of the $WW$, $WZ$, and $VZ$ are classified as background components, denoted as $VV$bkg. Acceptance uncertainties are summarised in Table \ref{tab:summary_altsamples} and detailed in the Appendix Table \ref{table:VV_Sys_Summary}. For the signal components, the uncertainties are split between the $ZZ$ and $WZ$ and include:
\begin{itemize}[leftmargin=*]
    \item \textit{Channel extrapolation uncertainties}: two sets are considered due to the differences between the components. One covers the 1L $\rightarrow$ 0L for $WZ$bb and $WZcc$, and the other the 2L $\rightarrow$ 0L for $ZZbb$ and $ZZcc$. They are split by \nj.
    \item \textit{Region extrapolation uncertainties}: from the \gls{sr} to the CRHigh and CRLow\cref{footnote-crlow}, due to the higher diboson purity of the \gls{sr}, with an additional \gls{sr} to CRLow in \vhb\ 1L. These uncertainties are separated for the different lepton channels.
    \item \textit{\nj\ acceptance}: are considered from 2-jet to higher jet-multiplicities. First to 3-jet, with a different value for the low \ptv\ region, then from 3-jet to 4-jet inclusively in \ptv for 0L and 2L. They are decorrelated between the different lepton channels.
    \item \textit{\ptv\ extrapolation uncertainties}: the 150 GeV < \ptv\ < 250 GeV region is the purest in signal diboson and is therefore used to extrapolate to the other \ptv\ regions, separately for the different lepton channels and \nj.
    \item \textit{\gls{stxs} binning acceptance uncertainties}: are included between \nj\ and \ptv\ regions for all $VZ$ signal processes. They are modelled by \gls{qcd} scale variations. 
\end{itemize}

For the background components of $WW$, $W_{\text{had}}Z_{\text{lep}}$, $W_{\text{lep}}Z_{\text{had}}$, and $ZZ$, where the ``had'' or ``lep'' index specifying the decay type of the bosons, the acceptances uncertainties are similar to those of the signal components and include:
\begin{itemize}[leftmargin=*]
    \item \textit{Channel extrapolation uncertainties}: 2 sets covering 1L $\rightarrow$ 0L (for $WW$ and $W_{\text{lep}}Z_{\text{had}}$) and 2L $\rightarrow$ 0L (for $ZZ$ and $W_{\text{had}}Z_{\text{lep}}$) are included due to difference in purities.
    \item \textit{Region extrapolation uncertainties}: from the \gls{sr} to the CRHigh, as to the diboson purity is higher in the \gls{sr}, separately for the different channels.
    \item \textit{Acceptance in jet multiplicity}: go from low (2-jet) to high jet-multiplicity. First to 3-jet, with a different value for the low \ptv\ < 150 GeV region. Then from 3-jet to 4-jet inclusively in \ptv\ for 0L and 2L. They are derived separately for the different lepton channels. 
    \item \textit{\ptv\ extrapolation uncertainties}: all extrapolation go from the 150 GeV < \ptv\ < 250 GeV region to the other \ptv\ regions, due to the higher purity in diboson of the medium \ptv\ range, separately for the different channels.
\end{itemize}

\paragraph{}In addition, the diboson processes are modelled with different shape uncertainties:
\begin{itemize}[leftmargin=*]
    \item \gls{carl} shape uncertainties comparing the nominal \textsc{Sherpa} 2.2.11 samples to the two alternative samples \textsc{Powheg}+\textsc{Pythia}8 and \textsc{Sherpa} 2.2.1. The former accounts for differences to the matrix element and parton shower, while the latter accounts for the mismodelled \ptv\ shape. These uncertainties are applied to all regions.
    \item \gls{qcd} scale shape uncertainties are included to model changes to the scales $\mu_R$ and $\mu_F$, similarly to the $V+$jets.
    \item \gls{pdf} shape uncertainties modelling variation to $\alpha_s$ are considered.
    \item \gls{ew} shape uncertainties are considered, similarly to $V+$jets.
\end{itemize}

\paragraph{Boosted regime} The modelling is similar to the resolved regime, with the uncertainties fully detailed in Table \ref{table:VV_SysBoos_Summary} of the Appendix. Small contributions from misidentified $W$ decays as jets or mis-reconstructed leptons are taken into account. The $ZZ$ and $WZ$ have normalisation uncertainties of 17\% and 27\% respectively. Acceptance uncertainties are included to cover the lepton channel acceptance, \ptv\ acceptance, and \gls{stxs} uncertainties on the \ptv\ and \nj\ bins, as is done in the resolved regime.

\subsection{Multi-jet Modelling}\label{sec-modMultiJ} 
The multi-jet background is negligible in 0L and 2L and in the boosted regime. In 1L, a data-driven estimate is used from a high-purity multi-jet control region obtained by inverting the lepton isolation requirements. Shapes are derived by a template fit on the $m_T^W$ distributions in the multi-jet \glspl{cr}. The shapes of the multi-jet are extracted to the \glspl{sr} of the resolved regime, primarily in \vhc, with extrapolation and normalisation uncertainties applied. Top and $W+$jets scale factors are applied to the template to account for the non-insignificant contributions of these processes in the multi-jet \glspl{cr}.