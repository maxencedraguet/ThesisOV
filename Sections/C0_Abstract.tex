
%\begin{abstract}
\vspace*{\fill}
\begin{center}
\textbf{\large \color{oxfordblue} ABSTRACT}
\end{center}
Identifying the flavour of jets plays an essential role in many ATLAS analyses. The outcome of the hadronisation of quarks and gluons, jets leave a rich signature of numerous particles emitted in a main direction from the initially decaying one. This subject is extensively discussed in this thesis, with a complete review of the algorithmic developments carried out by the ATLAS Collaboration from 2020 to early 2024. Increasingly sophisticated machine learning models called taggers have been developed for this specific purpose. The classical approach relies on a hierarchical construction combining low-level physically-motivated taggers with a Deep Set or a Recurrent Neural Network as inputs to a high-level network predicting the flavour. Recently, a more nimble design leveraging a single network to deliver state-of-the-art performance has been introduced. The core of this network is either a Graph Attention Network or a Transformer Encoder unit. Expert knowledge is passed to the model by optimising multiple tasks, with different physics input types analysed in a multimodal framework. The design and training of these taggers are reviewed, with a study of the hyperparameter optimisation for large networks using techniques from the ML literature on Large Language Models. \\

Following the 2012 discovery of the Higgs boson by the ATLAS and CMS Collaborations, increasingly refined measurements of the new particles have been performed. The leading production modes and the decay mode to third-generation fermions and gauge vector bosons of the Higgs have now all been measured. Attention is shifting to the second-generation fermions, such as the $c$-quark, and on precision differential cross-section measurements. This thesis presents a combined search for the $H \rightarrow c\bar{c}$ coupled with a differential measurement of the $H \rightarrow b\bar{b}$ in the $VH$ production mode. The analysis exploits the full 140 fb$^{-1}$ proton-proton collision luminosity collected in Run 2 by the ATLAS experiment at a centre-of-mass energy of 13 TeV. The combination of the decay modes allows for a coherent joint analysis strategy improving the constraining of the shared backgrounds. Flavour taggers are used to identify candidate $b$- and $c$-jets to reconstruct the Higgs. The full \pt\ spectrum is covered, with the two candidate jets resolved at low momentum and a single merged boosted signature at high momentum. Three leptonic channels are defined based on the number of electrons and muons found in the final state. A fine categorisation is deployed with dedicated Boosted Decision Trees signal discriminants to increase the sensitivity. The analysis is blinded with an expected 95\% CL$_s$ upper limit for the \vhc\ signal strength of 11.1 $\times$ the Standard Model prediction. The \vhb\ expected signal strength is 7.9~$\sigma$ over the background-only hypothesis, with the $WH$ and $ZH$ productions respectively measured with expected significances of 5.5~$\sigma$ and 6.2~$\sigma$. A standard cross-section template measurement is performed in stage 1.2 for \vhb, in bins of \pt\ and number of additional jets.
%\end{abstract}
\vspace*{\fill}
    